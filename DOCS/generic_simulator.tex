
\documentclass[a4paper,oneside,notitlepage]{article}
%%%%%%%%%%%%%%%%%%%%%%%%%%%%%%%%%%%%%%%%%%%%%%%%%%%%%%%%%%%%%%%%%%%%%%%%%%%%%%%%%%%%%%%%%%%%%%%%%%%%%%%%%%%%%%%%%%%%%%%%%%%%%%%%%%%%%%%%%%%%%%%%%%%%%%%%%%%%%%%%%%%%%%%%%%%%%%%%%%%%%%%%%%%%%%%%%%%%%%%%%%%%%%%%%%%%%%%%%%%%%%%%%%%%%%%%%%%%%%%%%%%%%%%%%%%%
\usepackage{amssymb}
\usepackage{amsmath}

\setcounter{MaxMatrixCols}{10}
%TCIDATA{OutputFilter=LATEX.DLL}
%TCIDATA{Version=5.00.0.2552}
%TCIDATA{<META NAME="SaveForMode" CONTENT="1">}
%TCIDATA{Created=Sunday, April 03, 2005 11:48:37}
%TCIDATA{LastRevised=Tuesday, July 26, 2005 18:32:22}
%TCIDATA{<META NAME="GraphicsSave" CONTENT="32">}
%TCIDATA{<META NAME="Title" CONTENT="Generic Simulator">}
%TCIDATA{<META NAME="DocumentShell" CONTENT="Standard LaTeX\Blank - Standard LaTeX Article">}
%TCIDATA{Language=American English}
%TCIDATA{CSTFile=40 LaTeX article.cst}

\setlength{\textheight}{8in}
\setlength{\textwidth}{6in}
\setlength{\oddsidemargin}{0mm}
\setlength{\evensidemargin}{0mm}
\setlength{\marginparwidth}{0mm}
\setlength{\marginparsep}{0mm}
\newtheorem{theorem}{Theorem}
\newtheorem{acknowledgement}[theorem]{Acknowledgement}
\newtheorem{algorithm}[theorem]{Algorithm}
\newtheorem{axiom}[theorem]{Axiom}
\newtheorem{case}[theorem]{Case}
\newtheorem{claim}[theorem]{Claim}
\newtheorem{conclusion}[theorem]{Conclusion}
\newtheorem{condition}[theorem]{Condition}
\newtheorem{conjecture}[theorem]{Conjecture}
\newtheorem{corollary}[theorem]{Corollary}
\newtheorem{criterion}[theorem]{Criterion}
\newtheorem{definition}[theorem]{Definition}
\newtheorem{example}[theorem]{Example}
\newtheorem{exercise}[theorem]{Exercise}
\newtheorem{lemma}[theorem]{Lemma}
\newtheorem{notation}[theorem]{Notation}
\newtheorem{problem}[theorem]{Problem}
\newtheorem{proposition}[theorem]{Proposition}
\newtheorem{remark}[theorem]{Remark}
\newtheorem{solution}[theorem]{Solution}
\newtheorem{summary}[theorem]{Summary}
\newenvironment{proof}[1][Proof]{\noindent\textbf{#1.} }{\ \rule{0.5em}{0.5em}}
\input{tcilatex}

\begin{document}

\title{Generic Simulator}
\author{Dimitriu Gabriel}
\maketitle
\tableofcontents

\section{\protect\bigskip Introduction}

Realization of modelation and simulation of one system is composed from
three base objects: \textit{the real system }which is seen like a the source
of data; \textit{the model,} which is a collection of instructions to
generate the comparable dates with the real system; \textit{the simulator,}
which is an algorithm programmed into one language which execute the models
instructions.

\subsection{Formal definition of base models}

Any simulation model could be considered like a real time dynamic system,
its components are described by three types of variables: input variables,
state variables and output variables. The dynamic behavioral of the model is
described by the DEVS (Discrete Event System Specification) object, this is
define:

\begin{equation}
DEVS=(X,S,Y,ta,\delta _{int},\delta _{ext},\lambda )  \label{DEVS}
\end{equation}

where:

\begin{itemize}
\item X is the "input" collection, Cartesian product of the input variables.

\item Y is the "output" collection, cartesian product of the output
variables.

\item S is the "state" collection, cartesian product of the state variables.

\item $ta:S\rightarrow R_{+}\cup (0,\infty )$ is the time advance
function,its value $ta(s),s\in S$ \ represents the time interval after the
system make a transition between state $s$ and state $\delta _{int}(s)$.

\item $\delta _{int}(s):S\rightarrow S$ is the internal transition function
of the system, this function represent the transition between one and
another state without the external influence.

\item $\delta _{ext}(s):Q\times I\rightarrow Q$, where $Q=\{(s,e)|s\in
S,e\in R,0\leq e\leq ta\}$ is the external transition function of the
system, this function represent the transition between one and another state
when an external event arrived at time $e$.

\item $\lambda :S\rightarrow Y$ is the output function, if the system is in
state s then $\lambda (s)$ will be the information sent to the output before
the systems make a transition given by the transition function (internal or
external).
\end{itemize}

\begin{definition}
The base model is the model ruled by DEVS formal definition.
\end{definition}

The system function is define by the following function:

\begin{equation*}
\left\{ 
\begin{array}{l}
\exists e\in \left[ T,T+ta(s)\right] \,\Rightarrow \,\exists x\in X\,\
si\,se\,executa\,\delta _{ext}(s,e,x) \\ 
\nexists e\in \left[ T,T+ta(s)\right] \,\Rightarrow \,\exists x\in X\,\
si\,se\,executa\,\delta _{int}(s)%
\end{array}%
\right.
\end{equation*}

\subsection{Network with static structure}

Coupling more models is done by using a new model, \textit{the network model}%
, which can be with static or dynamic structure.

The static structure model is define by%
\begin{equation}
DN=(D,\{M_{i}\},\{I_{i}\},\{Z_{i,j}\},select)  \label{DN}
\end{equation}

where

\begin{itemize}
\item D is identifications collection of system components.

\item for each $i\in D$ we have

\begin{itemize}
\item $M_{i}$ is the base model associated with i

\item $I_{i}$ is the collection of systems influence by i
\end{itemize}

\item for each $j\in I_{i}$ we have the function $Z_{i,j}$ which is the
message transfer function from the system \thinspace $i$ to the system $j$.

\item select is the selection function: from the components who change their
state in the same moment this function choose the component which influence
the state of the coupled model.
\end{itemize}

\begin{theorem}
Any model defined using the stucture (\ref{DN}) is a base model.
\end{theorem}

\section{Class Hierarchy and functionality}

Here there are three different types of classes:

\begin{enumerate}
\item Generic classes for the simulator data and functions. These classes
are inherited by other generic classes

\item Generic classes for the simulator. These classes are inherited by user
classes.

\item User classes. These classes are the simulator specific application.
\end{enumerate}

Type 1 classes are: \textit{CGData, CGRepartition, Queues, Cparser\_func,
Cparser\_func\_data, }\linebreak \textit{\ Cparser\_func\_memory.}

Type 2 classes are: \textit{CGInput, CGOutput, CGNetworkS,
CGProcessor,CGTransition, CGWire.}

Type 3 classes are: \textit{CInput, COutput, CNetworkS, CProcessor,
CTransition, CCoordonator}.

\subsection{Class CGData}

This class inherits the CObject class and is the base for the CGInput and
CGOutput classes.

This class basic functionality is to hold the data and time for transfer
(ports) and internal processor Quells.

Principal functions:

\begin{itemize}
\item CGData is the constructor who can take one or two arguments: first
argument is the name of the port who held the data and the second argument
is optionally and is the initial value of the data (the second argument is
used for quells of the processor if we want only the value).

\item GetDim is the function who return the dimension of the data port.

\item GetName is the function who return the name of the port.

\item SetDim is the function who set the dimension of the data port.

\item SetWTime is the function who set working time of the data who is
holded by the port.

\item SetTime is the function who set \ the time when the data was created.

\item SetParent is the function who set the parent who held this port.

\item GetParent is the function who return the parent who held this port.

\item SetParentW is the function who set the wire who held this port into
connection with other ports.

\item GetParentW is the function who return the wire who held this port.

\item \textit{GetClientExt} is the function who return the external client.

\item \textit{Set} is the function who set the data value.

\item \textit{Get} is the function who get the data value.

\item \textit{Reset} is the function who reset the data value.

\item \textit{SetSource} is the function who set the pointer to the parent
of the data.

\item \textit{GetSource} is the function who return the pointer to the
parent of data.
\end{itemize}

Functions \textit{Set,Get} and \textit{Reset} are virtual functions who are
replaced by function in \textit{CGInput} and \textit{CGOutput} classes.

\subsection{Class CGInput}

This class inherits the \textit{CGData} and it must be derived by class 
\textit{CInput} \ to obtain the correct input class.

This class is the input port for all systems: processor and network.

This class also held the pointer to the generator and file if the user
desired to have an log file of the generator.

If the dimension of port is multidimensional because this wires are
synchronized it can held a generator for each element of data.

Functions:

\begin{itemize}
\item \textit{CGInput} is the constructor for the input port. It has two
arguments: one is the name of the port and the second argument, who is
optionally, is the name of the file in which we put the log.

\item \textit{SetGenerator} is the function who set the type of the
generator for each element of the data if the dimension is greater than 1.

\item \textit{Set} is the function who set the value of data; this function
overrides the function \textit{Set} defined in \textit{CGData} class. First
version who takes no argument and who generate a value given by the
generator. Second version who takes one argument and who put the value given
by the argument into data; if the port has a generator the argument is added
to the generated by the generator. If a file is defined the value is also
put in the log file.
\end{itemize}

\subsection{Class CGOutput}

This class inherits the \textit{CGData} and it must be derived by class 
\textit{COutput} \ to obtain the correct input class.

This class is the output port for all systems: processor and network.

Functions:

\begin{itemize}
\item \textit{CGOutput} is the constructor. It has two argument: first
argument is the name of the port and the second argument, who is optionally,
is the name of the log file.

\item \textit{Set} is the function who set the value of the port. If a file
is defined the value is also put in the log file.
\end{itemize}

\subsection{Classes \textit{Cparser\_func, Cparser\_func\_data,
Cparser\_func\_memory}}

These classes are for parser for functions. These classes are called from
CGTransition class.

\subsection{\protect\bigskip Class CInput}

This class inherits \textit{CGInput}. This class have to be derived to
accommodate the specific function of the port.

\subsection{Class CGWire}

This class is the base of the communication of the networks. It has several
base functions: send the message from an input (output) port to the
connected output (input) port from diferse networks and processors; make teh
sum of different ports if they have the same dimensions.

Functions:

\begin{itemize}
\item \textit{CGWire} is the constructor who takes three arguments: first
argument is the pointer to the parent port (input/output) from which we take
the data and commands; second argument is the type of the wire: input,
output, internal; third argument is the pointer to the parent network who
has this wire.

\item \textit{Add} is the function who add a new port to the wire. This port
is connected to the parent port.

\item \textit{receivemsg} is the function who send the message thruout the
coonection.It has two arguments: first argument is the type of message
(init, internal, external or output); second argument is the time when the
message has arrived.

\item \textit{select} is the function who select the closest end of network
or processor.

\item \textit{sigma} is the function who return the minimum of end time or
the element with the lowest queue from the connected elements.

\item \textit{IsHere} return true is the port given in the argument is
connected in this wire.

\item \textit{GetSource} return the source of the connected port.
\end{itemize}

\subsection{Class CGNetworkS}

This is the base class who contain all the information. This class held the
connection between the processors and internal networks. This class is also
the frontend of the simulation.

Functions:

\begin{itemize}
\item \textit{CGNetworkS} is the constructor for the class. It has two
parameters: first which is optional hold the parent of the network, if this
is NULL or isn't given, this is conssider that that network is the frontend
of the simulation; the second argument is the name of the network.

\item \textit{GetName} is the function who return the name of the network.

\item \textit{SetParent} is the function who set the parent of the network,
it overwrite the parameter given by the constructor.

\item \textit{GetParent} is the function who return the parent of the
network.

\item \textit{GetPort} is the function who return the pointer and the type
in the second argument to the port characterisated by the name.

\item \textit{AddPort} is the function who add a port to a network and the
type of the port (input/output).

\item \textit{AddNetworkNetwork} is the function who add a wire between
ports of two internal networks.

\item \textit{AddProcNetwork} is the function who add a wire between ports
of one processor and one internal network.

\item \textit{AddNetworkProc} is the function who add a wire between ports
of one internal network and one processor.

\item \textit{AddParentNet} is the function who add a wire between a port
from the network and a port from an internal network.

\item \textit{AddParentProc} is the function who add a wire between a port
from the network and a port from an internal processor.

\item \textit{AddProcProc} is the function who add a wire between a port
from one internal processor to an proc from another internal processor.

\item \textit{AddNetwork} is the function who add the appartenence of a
network to this network.

\item \textit{AddProc} is the function who add the appartenence of a
processor to this network (this is use for active processors).

\item \textit{AddProcP} is the function who add the appartenence of a
processor to this network (this is use for passive processors).

\item \textit{GetTypeWire} is the function who return the type of the wire
(true for unique signal).

\item \textit{SetTypeWire} is the function who set the type of the wire
(true for unique signal).

\item \textit{GetInType} is the function who return the type of input data.

\item \textit{SetInType} is the function who set the type of input data
(false for data is not fractionary and true the data is divided thruout the
processors and networks).

\item \textit{SetTypeQueue} is the function who set the administration type
of queue (false for descentralized and true for centralized).

\item \textit{GetTypeQueue} is the function who return the administration
type of queue.

\item \textit{SetClockPrecision} if the function who set the precision of
the clock $\pm $\ the value when a process is conssider out or in.

\item \textit{PrintData} is the function who print the curent data on stdout.

\item \textit{sigma} if the function who return the value of the input port
(when the process is end).

\item \textit{SetProc} is the function who set the processor,given by it's
pointer, has executed the transition at a specific time.

\item \textit{SetNet} is the function who set \ the network, given by it's
pointer, has executed the transition at a specific time.

\item \textit{receivemsg} is the function of transfer messages:
0-init,1-internal,2-external,3-output.
\end{itemize}

\subsection{Class CGProcessor}

This class is the base class of the simulator. It contain the active
processor.

\begin{itemize}
\item \textit{AddPort} it adds a \ port to the processor if type is true it
is input port if it is false is output port.

\item \textit{GetPort} it returns the pointer to the port with name port and
return in type 0(is not found),1(is input port),2(is output port).

\item \textit{SetParent} it sets the network parent of the \ processor.

\item \textit{GetParent} it returns the pointer to the parent of the
processor.

\item \textit{EnableQ} is the function who enable the queue of the processor.

\item \textit{DisableQ} is the function who disable the queue of the
processor.

\item \textit{sigma} is the function who returns the done time and the
dimension of the queue.

\item \textit{SetClockPrecision} is the function who sets the increment of
clock, it must have the same value as the clock in the coordonator, the
simulation clock.

\item \textit{GetPrevTime} is the function who returns the value of the time
of the last event.

\item \textit{GetTime} is the function who return the current time of
simulation.

\item \textit{GetName} is the function who return the name of the processor.

\item \textit{CGProcessor} is the constructor which has some important
arguments: type who sed if in a period of working the processor hold the
data or it reset the outputs port to zero; generator is the transition
function of the processor.

\item \textit{receivemsg} is the interaction function of the processor. type
has 4 value: 0=initialization of the processor, 1=internal transition,
2=external transition and 3=get to the output the actual values.

\item \textit{SetFirst} is the function who set if the first transition
could be one internal of it must be one external.

\item \textit{IsDelayed} is the function who set if the data is delayed or
the value stored is the correct values.

\item \textit{SettN} is the function who set the new done time.

\item \textit{SettL} is the function who set the last event time.

\item \textit{receive0} is the function who is executed in case of
transition 0.

\item \textit{receive1} is the function who is executed in case of
transition 1.

\item \textit{receive2} is the function who is executed in case of
transition 2.

\item \textit{receive3} is the function who is executed in case of
transition 3.

\item \textit{set\_extfunc} is the function who set the external function,
so the function for transition 2.

\item \textit{set\_intfunc} is the function who set the internal function,
so the function for transition 1.
\end{itemize}

\subsection{CGTransition}

Is the class who held the real functionality of a processor. This class must
be derivate in all cases to achieve the real functionality of the processor.

\begin{itemize}
\item \textit{SetRepartitionT} is the function who set the repartition table
for an input repartion.

\item \textit{GetIndexT} is the function who return the real index in the
table of the input repartition for the argument.

\item \textit{Set} is the function who set the seed for the random number
generator.

\item \textit{GetIndexR} is the function who return

\item \textit{SetClk} is the function who set the clock of the simulation
for a waitting process.

\item \textit{SetParent} is the function who set the parent processor of the
transition.

\item \textit{CGTransition} is the constructor who create the actual random
repartition.

\item \textit{proc\_deltaext} is the function who is executed when an
external transition cames (it is virtual because real application must
derivate it).

\item \textit{proc\_delatint} is the function who is executed when an
internal transition cames (it is virtual because real application must
derivate it).

\item \textit{proc\_lambda} is the output function which is executed after
each transition (it is virtual and some application must derivate it if they
want to filter some output ports).

\item \textit{assign\_memory} is the function who assign the variable memory
for the parser for all function.

\item \textit{set\_extfunc} is the function who set the parser for the
external function for the specific output port.

\item \textit{set\_intfunc} is the function who set the parser for the
internal function for the specific output port.

\item \textit{is\_parallel} is the function who set the parser in parallel
mode, that mean all the wire from an output port with dimension great than 1
is conssider to have the same internal and external function.
\end{itemize}

\section{Specific implementation of classes}

There are the classes with specific application like: mux/demux
(DistributionBus), bus, etc

\subsection{CGDistributionBus}

Thsi is the class which implementate the MUX/DEMUX.

\begin{itemize}
\item \textit{GDistributionBus} is the constructor who takes two arguments:
one the name of the instance and second is the type who is parsed to the
processor.

\item \textit{SetType} is the function who set the type of the class:
DEMUX/MUX.

\item \textit{sigma} is the function who return the closest done time of the
element.

\item \textit{proc\_lambda} is the output function.

\item \textit{receivemsg} is the interaction function for transition.

\item \textit{receive0} is the function who is executed in case of
transition 0.

\item \textit{receive1} is the function who is executed in case of
transition 1.

\item \textit{receive2} is the function who is executed in case of
transition 2.

\item \textit{receive3} is the function who is executed in case of
transition 3.
\end{itemize}

\subsection{CGBus}

Is the class who distributes a multidimensional wire to other wires.

\begin{itemize}
\item \textit{CGBus} is the constructor who takes two parameters: one is the
name and the second is the type who is parsed to the processor base class.

\item \textit{SetSync} is the function who set if the element waits for all
input elements to comes and the transition is done at next internal
transition or when all input elements arrived the transition is done.

\item \textit{proc\_lambda} is the output function.

\item \textit{receivemsg} is the interaction function for transition.

\item \textit{receive0} is the function who is executed in case of
transition 0.

\item \textit{receive1} is the function who is executed in case of
transition 1.

\item \textit{receive2} is the function who is executed in case of
transition 2.

\item \textit{receive3} is the function who is executed in case of
transition 3.
\end{itemize}

\end{document}
